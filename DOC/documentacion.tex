%Sobre el documento
\documentclass[12pt]{article}
\pagestyle{empty}
\setlength{\parindent}{12pt}

%Importar paquetes
\usepackage{float}
\usepackage{amsfonts, amssymb, amsmath}
\usepackage{enumerate}
\usepackage[T1]{fontenc} %Allows to works with ñ

%Informaciòn sobre el
\title{Documentation about diario personal}
\author{Frutos, I\~{n}aki}
\date{\today}

%Comienza el texto
\begin{document}
\maketitle
\tableofcontents

\section{Objetivos del programa}
\begin{normalsize}
 El objetivo de este programa es crear un server, con usuarios 	y contraseñas, que sirva para guardar información acerca de lo que se realizó en ese día, con archivos de texto, imagenes y 	links a guardar
 \end{normalsize}
\section{Tiempo estimado del proyecto}
	El tiempo que va a tardar este proyecto va a ser de unas 3 			semanas aproximadamente
\section{Cosas por hacer}
	\begin{enumerate}
		\item Crear archivo HTML del mes
		\item Crear archivo HTML de la semana
		\item Crear archivo HTML del día
		\item Crear el server en JS
		\begin{enumerate}
			\item Hacer todas las rutas
			\item Hacer la parte de las contraseñas
			\item Buscar sobre como funciona el hashing y hacer para evitar que se pueda robar las contraseñas
			\item Pensar, diseñar e implementar el JSON a enviar al HTML necesiario para completar el calendario
		\end{enumerate}
		\item Crear el CSS
		\item Crear el js del lado del cliente para administrar bien la informacion mandada
	\end{enumerate}
\section{Colores del CSS}
La paleta de colores será la siguiente
\begin{enumerate}
    \item FFE9D0
    \item FFFED3
    \item BBE9FF
    \item B1AFFF
\end{enumerate}
\section{Documentacion sobre el programa}
	
	
\end{document}